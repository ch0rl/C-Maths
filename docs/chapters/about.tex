\chapter{About}
    \section{What is this?}
        This is a simple, command-line, tool to parse mathematical expressions. 

        \subsection*{Features}
            In the following:
            \begin{itemize}
                \item '*' indicates it is not yet implemented.
                \item \textcolor{red}{Inputs are red}, \textcolor{blue}{Outputs are blue}.
            \end{itemize}

            \noindent This program provides the following features:
            \begin{itemize}
                \item Evaluating expressions (ie., \textcolor{red}{$2 * (4 + 9)$} \textcolor{blue}{$= 26$}).
                \item[*] Comparing expressions (ie., \textcolor{red}{$2 * (a + 3) = 2 * a + 6$} \textcolor{blue}{true}).
                \item[*] Symbol assignment and memory (ie., \textcolor{red}{$a = 6$; $2 * a$} \textcolor{blue}{$= 12$}).
                \item[*] \LaTeX support (ie., \textcolor{red}{\mintinline{text}{3 \cdot 9}} \textcolor{blue}{$= 27$}) 
            \end{itemize}

    \section{But why?}
        Because I wanted to.

    \section{Capabilities}
        As of \today{}, the program is capable of the following.

        \subsection*{Tokenising}
            \begin{tabular}{p{.3\textwidth} | p{.65\textwidth}}
                Type & Regex\footnotemark{} \\\hline
                Integers & \mintinline{text}{/-{0,1}[0-9]*\.{0,1}[0-9]+/} \\
                Symbols & \mintinline{text}{/[a-zA-Z]+/} \\
                Operators & \mintinline{text}{/[+\-/*^=]/} \\
                Braces &  \\
            \end{tabular}\footnotetext{This program doesn't use regexes but it is an easy way to describe strings.}

        \subsection*{Evaluating}
            \begin{itemize}
                \item Operators whose children are all numeric.
            \end{itemize}
