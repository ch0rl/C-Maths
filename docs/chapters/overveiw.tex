\chapter{Overview}
    This chapter will focus on an overview of how the program functions and is split into sections based on the directory listing:
    \dirtree{%
        .1 C-Maths.
        .2 Errors.h.
        .2 Evaluate.h.
        .2 Functions.h.
        .2 Maths.c.
        .2 Structures.h.
        .2 Symbols.h.
        .2 Tokenise.h.
        .2 ToTree.h.
    }

    \section{Maths.c}
        `Maths.c' is the main program that draws everything together. The header files can be used in isolation and this file simply provides an interface with the following.

    \section{The Main Parts}
        \subsection{Tokenising}
            The first step in interpreting maths is to tokenise the string; this is done in `Tokeniser.h'. At a high level, this works by iterating over the string and forming `tokens' from the characters/sub-strings therein. Token types used in this program are:
            \begin{itemize}
                \item Null (\mlc{T_NULL})
                \item Functions (\mlc{T_FUNC})
                \item Operators (\mlc{T_OP})
                \item Symbols\footnote{`Symbols' in this program are akin to variables. For example, in $n_a + 7 = y$, $n_a$ and $y$ are symbols.} (\mlc{T_SYM})
                \item Open braces (\mlc{T_OPEN})
                \item Close braces (\mlc{T_CLOSE})
                \item Numbers (\mlc{T_NUM})
            \end{itemize}

        \subsection{Converting to a Tree}
            To parse the expression, this program uses an abstract syntax tree\footnote{See: \url{https://en.wikipedia.org/wiki/Abstract_syntax_tree}.}. To form this form an array of tokens, the program uses:
            \begin{enumerate}
                \item The Shunting Yard Algorithm, then
                \item A stack-based algorithm.
            \end{enumerate}

            \subsubsection*{Shunting Yard}
                The Shunting Yard Algorithm\footnote{See: \url{https://en.wikipedia.org/wiki/Shunting_yard_algorithm}.} is used to convert an `infix' expression to a `postfix' expression. 

                Infix is the common form we use to express maths -- with an operator between the operands; postfix, however, is easier for programs to parse and has the operator after the operand. For example, the infix expression $2 \cdot ( 3 + a )$ would be equivalent to the postfix $3~a~+~2~\cdot$.
            
            \subsubsection*{Forming The Tree}
                Once the tokens are in postfix form, the tree can be formed. This is done through a simple algorithm that iterates over the tokens and:
                \begin{itemize}
                    \item If the token is an operand (number or symbol), a node with the same value is pushed to the stack.
                    \item If the token is an operator, nodes are popped from the stack (the amount of nodes is determined by how many operands the operator expects) and a new node is pushed with the value being the operator, and the children being the operands.
                \end{itemize}

                \noindent Once complete, the last (and only) value on the stack is the head of the tree.
        
        \subsection{Evaluation}
            Evaluating the tree is the (in my opinion, at least) most important part of this program -- and the part I did without consulting any normal standards; the following algorithms I made up myself. This part works in 5 main ``rounds''.

            \subsubsection{Round 1 -- Replacing Symbols}
                To begin, the tree is traversed and -- consulting the symbol table -- symbols are replaced with their respective value. This value will only exist if it has been set in a previous input. For example, the user enters $a = 5 + b$ and then $a \cdot 2$; after this round, the tree would be equivalent to $(5 + b) \cdot 2$.

            \subsubsection{Round 2 -- Numeric Evaluation}
                The second round evaluates functions/operators whose children are all numeric. For example, this would parse $3 + 9$ but not $a - 2$. 

            \subsubsection{Round 3+}
                To be implemented.

    \section{Misc.}
        \subsection{Errors.h}
            `Errors.h' provides functions for when a program produces a fatal error. For example, `\mlc{err_red_location}' which outputs the location and error passed (with colour-based formatting) and then exits. 

        \subsection{Structures.h}
            `Structures.h' provides the majority of structures used throughout, as well as the necessary accompanying functions. Such structures include:
            \begin{itemize}
                \item Tree nodes,
                \item Stacks,
                \item Queues, and
                \item Symbols.
            \end{itemize}
